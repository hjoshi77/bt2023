\documentclass{exam}

\begin{document}

\begin{center}
  \bfseries\large
  Department of Bitechnology\\
  Biostatistics  BT2023\\
  Fall semester Aug-Dec 2022\\
  Quiz 1\\
  6 September 2022

\begin{flushright} \textbf{Maximum Marks: 20} \end{flushright} 

  \bigskip
\fbox{\fbox{\parbox{5.5in}{\centering
If required, justify the method used in finding the answer and its limitation}}}
\end{center}
\begin{flushleft} \textbf{Maximum Time: 30 mintuts} \end{flushleft}  
\vspace{5mm}

\begin{questions}
	\question The systolic blood pressures of a group of women were measured (in mmHg) after administration of a newly developed oral contraceptive.\\
\newline
127, 128, 140, 119, 145, 130, 148, 135, 129, 137, 128, \\
133, 139, 121, 137, 131, 120, 125, 137, 127, 138, 122 \\

Find out the \textbf{mean, mode, median and interquartile range} of the blood pressure\\
\question In order to generate a histogram in a given frequency distribution, what is the best way to define the number of bins. In other words, what is the formula which gives a nice estimate of the number of bins in a histogram ?
\question Write down the empirical relationship between arithmecal mean, mode and median. 
\question Compute the average burned calories alongwith the \textbf{standard deviation} for the following data set and  find out the \textbf{coefficient of the skewness}.\\
\newline
\begin{tabular}{ |c|c|c|c|c|c|c| } 
 \hline
 Kcal & 100-129 & 130-159 & 160-189 & 190-219 & 220 -249 \\ 
\hline
 Frequency & 5 & 4 & 12 & 6 & 3 \\ 
 \hline
\end{tabular}
\end{questions}
\end{document}
